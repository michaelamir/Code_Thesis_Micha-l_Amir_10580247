

\definecolor{titlepagecolor}{cmyk}{1,.60,0,.40}

\backgroundsetup{
scale=1,
angle=0,
opacity=1,
contents={\begin{tikzpicture}[remember picture,overlay]
 \path [fill=titlepagecolor] (current page.west)rectangle (current page.north east); 
 \draw [color=white, very thick] (5,0)--(5,0.5\paperheight);
\end{tikzpicture}}
}

\makeatletter                   
\def\printauthor{%                  
    {\large \@author}}          
\makeatother

\author{%
    Author 1 name \\
    Department name \\
    \texttt{email1@example.com}\vspace{40pt} \\
    Author 2 name \\
    Department name \\
    \texttt{email2@example.com}
    }



\begin{titlepage}
\BgThispage
\newgeometry{left=1cm,right=6cm,bottom=2cm}
\vspace*{0.4\textheight}
\noindent
\textcolor{white}{\Huge\textbf{\textsf{Hardy's Theorem}}}
\vspace*{2cm}\par
\noindent
\begin{minipage}{0.35\linewidth}
    \begin{flushright}
        \printauthor
    \end{flushright}
\end{minipage} \hspace{15pt}
%
\begin{minipage}{0.02\linewidth}
    \rule{1pt}{175pt}
\end{minipage} \hspace{-10pt}
%
\begin{minipage}{0.63\linewidth}
\vspace{5pt}
    \begin{abstract} 
An abstract is a brief summary of a research article, thesis, review, conference proceeding or any in-depth analysis of a particular subject or discipline, and is often used to help the reader quickly ascertain the paper's purpose. When used, an abstract always appears at the beginning of a manuscript, acting as the point-of-entry for any given scientific paper or patent application. Abstracting and indexing services for various academic disciplines are aimed at compiling a body of literature for that particular subject.
    \end{abstract}
\end{minipage}
\end{titlepage}
\restoregeometry
\lipsum[1-2]
\end{document}