\newpage
\section{Factoren en Invloed}
\label{h5}



\subsection*{Deelvraag 2: Welke factoren kunnen per bevolkingsgroep van invloed zijn op het wel of niet behalen van het maximum aantal kandidaten dat in de Tweede Kamer gekozen kan worden?}
Voor het behalen van het maximum per bevolkingsgroep zijn er een aantal factoren die van invloed kunnen zijn. Deze factoren kunnen invloed uitoefenen op mate van een succes van een door een bevolkingsgroep gekozen strategie. Alvorens we per specifieke bevolkingsgroep de factoren en de invloed van deze factoren gaan beschrijven, zullen we eerst de algemene factoren beschrijven die van invloed (kunnen) zijn op elke bevolkingsgroep.

\subsubsection*{Algemene factoren.}
De algemene factoren zijn de factoren die van invloed kunnen zijn op elke bevolkingsgroep. In deze sectie worden deze algemene factoren kort beschreven alvorens in de volgende secties er per specifieke bevolkingsgroep dieper op de factoren wordt ingegaan. De algemene factoren zijn:\\

\newcolumntype{L}{@{}>{\bfseries}p{14em}<{}}% Item label
\newcolumntype{I}{X@{}}% Item contents
\noindent\begin{tabularx}{\textwidth}{LI}
Aantal kandidaten: &  Voor het kunnen toepassen van één van de in het vorige hoofdstuk toegepaste strategie\"{e}n, is het in eerste instantie essentieel dat er \"{u}berhaupt kandidaten van een bevolkingsgroep op de kandidatenlijsten staan. Daarnaast is het aantal kandidaten van de belang om bijvoorbeeld een volledig door één bevolkingsgroep bezette Tweede Kamer te vormen. In de volgende secties betreffende de factoren per bevolkingsgroep wordt hier dieper op ingegaan. \\
\\  
Committeren aan strategie:  & In het vorige hoofdstuk is telkens per bevolkingsgroep onderzocht wat het theoretisch maximum is wanneer 100\% van de kiesgerechtigden zich committeert aan een strategie. Echter is het de vraag wat er gebeurt als een bepaald percentage van een bevolkingsgroep zich committeert aan een strategie. Waar het theoretisch maximum uitgaat van 100\% deelname van de leden van een bevolkingsgroep is dit in de praktijk moeilijk haalbaar. Het is daarom interessant te onderzoeken wat het theoretisch maximum per bevolkingsgroep is wanneer de meeste succesvolle strategie uit het vorige hoofdstuk wordt uitgevoerd door een deel van de bevolkingsgroep.  \\
\\
Keuze strategie: &  Een keuze van een strategie kan van invloed zijn op het wel of niet behalen van het theoretisch maximum. Een strategie kan moeilijk te communiceren zijn binnen een bevolkingsgroep en gemakkelijk te manipuleren zijn door 'tegenstanders'.\\
\\
\end{tabularx}  
 
 \newcolumntype{L}{@{}>{\bfseries}p{14em}<{}}% Item label
\newcolumntype{I}{X@{}}% Item contents
\noindent\begin{tabularx}{\textwidth}{LI}

Tegenstrategie: & Naast het feit dat de keuze van een bepaalde strategie van invloed kan zijn op het wel of niet behalen van het theoretisch maximum is het de vraag of een bevolkingsgroep met tegengestelde eigenschappen (bijvoorbeeld de bevolkingsgroep mannen tegenover vrouwen) een tegenstrategie zouden kunnen toepassen om te verhinderen dat tegengestelden in hogere mate vertegenwoordigd worden in de Tweede Kamer. \\
\\
 \end{tabularx}  


\paragraaf{}


\subsection{Factoren bevolkingsgroep vrouwen.}
\label{percV}

\paragraph{Aantal Vrouwelijke Kandidaten.}
Zoals in het vorige hoofdstuk al is beschreven was het bij de Tweede Kamerverkiezingen van 2012 onmogelijk om een Tweede Kamer te vorm die volledig werd bezet door vrouwelijke kamerleden. De SGP had helemaal geen vrouwelijke kandidaten op de kandidatenlijst en de PVV en de VVD hadden een hoger aantal zetels ontvangen dan het aantal vrouwelijke kandidaten dat deze partijen op de kandidatenlijsten hadden. Echter waren er bij de partijen die zetels hebben ontvangen toch het totaal van 179 vrouwelijke kandidaten waarop gestemd kon worden(zie Figuur \ref{fig:aaKandidaten} in Hoofdstuk  \ref{sec:math}). 

\paragraph{Committeren aan een strategie.}
Sinds de dagen van Aletta Jacobs en het ontstaan van de \textit{Eerste Feministische Golf} \citep{braun1992prijs}, de \textit{Dolle Mina's} en de \textit{Man Vrouw Maatschappij} in de jaren '60 en het ontstaan van de \textit{Tweede Feministische Golf} \citep{van2005vrouw} zijn er in de loop der jaren tal van personen en organisaties in Nederland bij gekomen die de belangen van vrouwen behartigen. Een kleine greep uit Nederlandse vrouwenbelangen organisaties levert op: de Nederlandse Bond voor Vrouwenkiesrecht, Nederlandse Unie voor Vrouwenbelangen, de Nederlandse Vrouwenraad, FNV Vrouw, Vereniging Vrouw en Recht etc. Daarnaast is minister Bussemaker van Onderwijs, Cultuur en Wetenschap \citeyearpar{navigerennaardetop} campagne begonnen die ervoor moet zorgen dat er meer vrouwen in topfuncties komen in Nederland. Volgens \cite{van2012tweede} krijgen met name de hoogstgeplaatste vrouwelijke kandidaten op de kandidatenlijsten een voorkeurstemmen van de vrouwelijke kiezer vanwege het feit dat de kandidaat van hetzelfde geslacht is. Tevens is het zo dat de hoogstgeplaatste vrouwelijke kandidaten op de kandidatenlijsten (de vrouwelijke lijsttrekkers buiten beschouwing gelaten) een hoog aantal voorkeurstemmen ontvangen. In Tabel \ref{table:HoogstV} is te zien dat zeven hoogstgeplaatste vrouwelijke kandidaten bij de offici\"{e}le verkiezingsuitslag genoeg stemmen hebben ontvangen om boven de voorkeursdrempel uit te komen. Het gemiddelde bij de hoogstgeplaatste vrouwelijke kandidaten ligt maar liefst op het aantal van 66.419 stemmen. Er kan daarmee worden aangenomen dat er een zekere animo bestaat bij Nederlandse vrouwen om zich te committeren aan een strategie die ervoor moet zorgen dat vrouwen in hogere mate vertegenwoordigd worden in de Tweede Kamer. 


\begin{table}[H]
\centering
	\begin{footnotesize}
		\begin{tabular}{llrr}
\toprule
{} &                 Partij &  Plaats op Lijst &  Aantal Stemmen  \\
Kandidaat                                &                        &                  &                           \\
\midrule
M.H.H. (Martine) Baay-Timmerman          &                 50PLUS &                3 &                      7123 \\
M.C.G. (Mona) Keijzer                    &                    CDA &                2 &                    127446 \\
C.J. (Carola) Schouten                   &           ChristenUnie &                3 &                     16507 \\
S. (Stientje) van Veldhoven-van der Meer &                    D66 &                2 &                     71170 \\
L. (Liesbeth) van Tongeren               &             GROENLINKS &                3 &                     10205 \\
E. (Esther) Ouwehand                     &  Partij v d Dieren &                2 &                     11573 \\
J. (Jetta) Klijnsma                      &                   PVDA &                2 &                    192190 \\
M. (Fleur) Agema                         &                    PVV &                2 &                     34943 \\
R.M. (Renske) Leijten                    &                     SP &                2 &                     69146 \\
E.I. (Edith) Schippers                   &                    VVD &                2 &                    123889 \\
\bottomrule
\end{tabular}

	\end{footnotesize}
			\caption{Het aantal stemmen dat de hoogstgeplaatste vrouwelijke kandidaten hebben ontvangen volgens de offci\"{e}le einduitslag. De SGP had geen vrouwelijke kandidaten.}
\label{table:HoogstV} 
\end{table} 

\indent Een deelname van 100\% van de vrouwelijke kiezers aan strategie 1 zou in theorie een Tweede Kamer op hebben kunnen leveren waarin 121 vrouwen plaats zouden nemen (zie Sectie \ref{vrouwen}. Echter is het de vraag wat er gebeurt wanneer een niet de volle 100\% van de vrouwelijke kiezers zich committeert aan de strategie maar een 'slechts' een deel van de vrouwelijke kiezers. In Figuur \ref{fig:PerV} is te zien hoeveel zetels er aan vrouwelijk kandidaten bedeeld zouden zijn wanneer een bepaald percentage van de vrouwelijke kiezers zich had gecommitteerd aan strategie 1. Hierbij moet genoteerd worden dat het aantal stemmen van vrouwelijke kiezers op een partij een percentage is van het totaal door vrouwelijke kiezers uitgebrachte stemmen (100\% deelname van vrouwelijke kiezers). Tevens nemen we aan dat het percentage van de vrouwelijke kiezers die zich committeren aan een strategie voor elke partij hetzelfde is. Oftewel wanneer bijvoorbeeld 25\% van de vrouwelijke kiezers zich committeren aan de strategie dan is geldt de 25\% voor elke partij. Ter illustratie het nemen we de VVD als voorbeeld.\\
\indent De VVD had volgens de einduitslag van de Tweede Kamerverkiezingen van 2012 een totaal van 1.077.127 vrouwelijke stemmen ontvangen (zie Figuur \ref{table:tab2V} in Sectie \ref{vrouwen}). Wanneer 25\% van de vrouwelijke kiezers zich committeren aan strategie 1 levert de strategie bij de VVD het aantal van ($25\%*1.077.127$ = ) 269.281 vrouwelijke stemmen stemmen op. De overige stemmen van vrouwelijke kiezers op de VVD laten we buiten beschouwing vanwege het feit dat we niet kunnen voorspellen naar welke partij of kandidaat deze stemmen gaan. Derhalve kunnen we alleen het aantal stemmen in ogenschouw nemen wat zekerheid biedt voor het berekenen van het aantal stemmen dat vrouwelijke kandidaten van vrouwelijk kiezers zouden gaan ontvangen om zodoende vrouwelijke kandidaten boven de kiesdrempel te helpen.  



\begin{figure}[H]


	\includegraphics[width=\linewidth]{percentages_van_vrouwen.png}

			\caption{Grafiek met overzicht van het maximum aantal zetels wat behaald kon worden wanneer een bepaald percentage van de vrouwelijke kiezers zich committeerde aan strategie 1.}

\label{fig:PerV}
\end{figure}

In Figuur \ref{fig:PerV} is te zien dat wanneer slechts 70\% van de vrouwelijke kiezers zich had gecommitteerd aan strategie 1, het theoretisch maximum nog altijd werd behaald. Bij 60\% van de vrouwelijke kiezers viel er één enkele vrouwelijke kandidaat af en bij slechts 40\% van de vrouwelijke kiezers zijn er nog altijd tien vrouwen meer in de Tweede Kamer dan daadwerkelijk bij de offci\"{e}le einduitslag het geval was (58 vrouwen in de Tweede Kamer na de offci\"{e}le einduitslag). 

\paragraph{Keuze Strategie.}
Wat betreft het kiezen van een strategie is strategie 1 de strategie die het meeste succes opleverde bij de bevolkingsgroep vrouwen. Zoals we weten, leverde strategie 1 het theoretisch maximum van 121 vrouwen in de Tweede Kamer. Echter levert strategie 2 ook een hoog rendement met 116 vrouwen in de Tweede Kamer. 

\paragraph{Tegenstrategie.}
Uit een onderzoek van \cite{lawless2012men} blijkt dat mannen in de Verenigde Staten het eigen geslacht geschikter achten voor het bekleden van een politieke ambt dan het vrouwelijk geslacht. Een vergelijkbaar onderzoek betreffende mannen en vrouwen in Nederland ontbreekt echter. Hoewel een onderzoek van \cite{mcelroy2009candidate} aangeeft dat kiezers in landen waar verkiezingen met een voorkeursstemregel niet discrimineren op basis van geslacht, is het mogelijk dat mannen het eigen geslacht prefereren als het gaat om het kiezen van vertegenwoordigers in de Tweede Kamer. Zodoende is het niet ondenkbaar dat mannen een tegenstrategie zullen toepassen om te voorkomen dat de vrouwen in hogere mate vertegenwoordigd woorden dan eerder het geval is geweest. 

\subsection{Factoren bevolkingsgroep allochtonen.}

\paragraph{Aantal allochtone Kandidaten.} 
Zoals in het vorige hoofdstuk al is beschreven was het bij de Tweede Kamerverkiezingen van 2012 onmogelijk om een Tweede Kamer te vorm die volledig werd bezet door allochtone politici. De SGP en de Partij voor de Dieren hadden geen enkele allochtoon op de kandidatenlijsten staan en, op 50PLUS en GROENLINKS na, ontvingen alle partijen bij de einduitslag meer zetels dan het aantal allochtonen dat zij op de kandidatenlijsten hadden staan. In totaal stonden er, bij de partijen die zetels hebben ontvangen, vijftig allochtone kandidaten op de kandidatenlijsten (zie Figuur \ref{fig:aaKandidaten} in Hoofdstuk \ref{sec:meth}). 


\paragraph{Committeren aan een strategie.}
Uit onderzoek van \citep{fennema2001civic} blijkt dat met name allochtonen van Turkse en Marokkaanse afkomst geneigd zijn op een kandidaat te stemmen met dezelfde afkomst. Echter lijken allochtone kiezers van Turkse afkomst in hogere mate te mobiliseren als het gaat om het uitbrengen van een stem bij de Tweede Kamerverkiezingen dan allochtonen van Marokkaanse afkomst. Onderzoek betreffende allochtonen met een andere afkomst dan een Turkse of Marokkaanse ontbreekt echter. Als het gaat om verbondenheid tussen etnische minderheden voelen personen behorende tot verschillende etnische minderheden zich in sterkere mate met elkaar verbonden wanneer zij zich benadeeld of minder geaccepteerd voelen \citep{schmitt2002meaning}. Volgens \cite{buijs2006strijders} voelen veel allochtonen in Nederland, van met name Turkse en Marokkaanse achtergrond, zich niet geaccepteerd. Volgens \cite{van2012tweede} geldt ook voor allochtonen dat zij in de meeste voorkeurstemmen uitbrengen op de hoogstgeplaatste allochtone kandidaat op de kandidatenlijst. In Tabel \ref{table:HoogstA} is te zien dat van de hoogstgeplaatste allochtone kandidaten op de kandidatenlijsten er slechts één allochtone kandidaat, in de persoon van Tanja Jadnanansing, boven de voorkeursdrempel uit is gekomen. Of dit ligt aan het feit dat zij stemmen van allochtone kiezers heeft gekregen, stemmen van vrouwelijke kiezers heeft gekregen of een combinatie ven beiden is niet duidelijk. Het gemiddelde bij de hoogstgeplaatste allochtone kandidaten lag op het aantal van 7.420 stemmen. Afgaande op het het aantal stemmen op de hoogstgeplaatste allochtone kandidaat lijkt de stelling van \cite{van2012tweede} niet helemaal op te gaan. Echter kan uit de eerdere stelling in deze paragraaf wel worden afgeleid dat er animo kan bestaan onder allochtonen om zich te committeren aan een strategie die ervoor moet zorgen dat allochtonen in hogere mate vertegenwoordigd worden in de Tweede Kamer. \\

\begin{table}[H]
\centering
	\begin{footnotesize}
		\begin{tabular}{lllrr}
\toprule
{} & Geslacht &        Partij &  Plaats op Lijst &  Aantal Stemmen \\
Kandidaat                 &          &               &                  &                           \\
\midrule
K.L.R. (Roy) Ho Ten Soeng &        M &        50PLUS &                6 &                      2654 \\
M. (Mustafa) Amhaouch     &        M &           CDA &               16 &                      1919 \\
I.S. (Ixora) Balootje     &        V &  ChristenUnie &                9 &                      2111 \\
V.A. (Vera) Bergkamp      &        V &           D66 &                5 &                     15387 \\
J.F. (Jesse) Klaver       &        M &    GROENLINKS &                4 &                      3351 \\
T.M. (Tanja) Jadnanansing &        V &          PVDA &                4 &                     28704 \\
J.D. (Jenny) Zerfowski    &        V &           PVV &               37 &                       210 \\
S. (Sadet) Karabulut      &        V &            SP &                6 &                     10572 \\
M. (Malik) Azmani         &        M &           VVD &               20 &                      1874 \\
\bottomrule
\end{tabular}

	\end{footnotesize}
			\caption{Het aantal stemmen dat de hoogstgeplaatste allochtone kandidaten hebben ontvangen volgens de offci\"{e}le einduitslag.}
\label{table:HoogstA} 
\end{table}





\indent Een deelname van 100\% van de allochtone kiezers aan strategie 1 zou in theorie een Tweede Kamer hebben opgeleverd waarin 34 allochtonen plaatsen zouden nemen (zie Sectie \ref{allochtonen}). Ook bij de bevolkingsgroep allochtonen gaan we onderzoek wat er gebeurt wanneer een deel van de bevolkingsgroep zich committeert aan strategie 1. Op dezelfde wijze als bij de bevolkingsgroep vrouwen is gedaan, gaan we berekenen hoeveel zetels er aan allochtone kandidaten worden bedeeld wanneer bepaalde percentages van de bevolkingsgroep allochtonen zich committeert aan de strategie (zie Sectie \ref{percV} voor uitleg van het bereken van het aantal stemmen). In Figuur\ref{fig:PerA} is te zien hoeveel zetels er aan allochtone kandidaten zouden zijn bedeeld wanneer een bepaald percentage van de allochtone kiezers zich had gecommitteerd aan strategie 1.  

\begin{figure}[H]


	\includegraphics[width=\linewidth]{percentages_van_allochtonen.png}

			\caption{Grafiek met overzicht van het maximum aantal zetels wat behaald kon worden wanneer een bepaald percentage van de allochtone kiezers zich committeerde aan strategie 1 of strategie 2. De Partij voor de Dieren en de SGP hadden geen allochtone kandidaten.}

\label{fig:PerA}
\end{figure}

In Figuur \ref{fig:PerA} is te zien dat wanneer slechts 60\% van de allochtone kiezers zich had gecommitteerd aan strategie 1, het theoretisch maximum nog altijd werd behaald. Bij 40\% van de vrouwelijke kiezers vielen er vijf allochtone kandidaten af en bij slechts 25\% van de allochtone kiezers zijn er nog altijd vier allochtonen meer in de Tweede Kamer dan daadwerkelijk bij de offci\"{e}le einduitslag het geval was (dertien allochtonen in de Tweede Kamer na de offci\"{e}le einduitslag). 


\paragraph{Keuze Strategie.}
Wat betreft het kiezen van een strategie maakt het voor de bevolkingsgroep allochtonen niet uit of strategie 1 of strategie 2 wordt uitgevoerd wanneer de volle 100\% van de allochtone kiezers zich committeren aan de strategie. Ook bij het uitvoeren van strategie 2 waarbij slechts een deel van de allochtone kiezers zich committeert aan de strategie is het resultaat hetzelfde. Het maakt niet geen verschil of er voor strategie 1 of strategie 2 wordt gekozen. 


\paragraph{Tegenstrategie.}
Volgens een artikel van \cite{de2000political} blijkt het zo te zijn dat rechtse partijen in zowel Nederland als Vlaanderen (bijvoorbeeld het Vlaams Blok) kiezers voor zich weten te winnen op basis van een negatief beeld dat deze kiezers hebben als het gaat om allochtonen- en migrantenkwesties. Daarnaast speelt ook chauvinisme een rol als het gaat om het stemmen op autochtone kandidaten en populistische partijen \citep{van2010swaying}. Hierdoor is het niet ondenkbaar dat een groep autochtone kiezers een tegenstrategie zal uitvoeren om juist meer autochtone kandidaten in de Tweede Kamer te krijgen.

\subsection{Factoren bevolkingsgroep ouderen.}
\label{percO}

\paragraph{Aantal oudere Kandidaten.} 
Zoals in het vorige hoofdstuk al is beschreven was het bij de Tweede Kamerverkiezingen van 2012 onmogelijk om een Tweede Kamer te vorm die volledig werd bezet door oudere politici. Het CDA, de PVDA, de VVD, de SP, D66 en de SGP ontvingen meer zetels dan dat zij oudere kandidaten op de kandidatenlijst hadden staan. In totaal stonden er, bij de partijen die zetels hebben ontvangen, 135 oudere kandidaten op de kandidatenlijsten (zie Figuur \ref{fig:jaKandidaten} in Hoofdstuk \ref{sec:math})

\paragraph{Committeren aan een strategie.}
In de literatuur bestaat er geen onderzoek betreffende voorkeurstemmen op oudere kandidaten. Echter kan aan de hand van de verkiezingseinduitslag worden opgemaakt dat er weinig sentimenten leven onder ouderen om beter vertegenwoordigd te worden door oudere politici in de Tweede Kamer. Een partij die specifiek opkomt voor belangen van ouderen is 50PLUS. Het feit dat 50PLUS slechts 4\% van de stemmen van ouderen heeft gekregen (zie Figuur \ref{fig:zetelsO} in Sectie \ref{ouderen}) suggereert dat ouderen ook andere aandachtspunten dan het zijn van een oudere belangrijk vinden als het gaat om het uitbrengen van een stem. Tevens is het zo dat de hoogstgeplaatste ouderen op de kandidatenlijsten (de lijsttrekkers buiten beschouwing gelaten) over het algemeen weinig voorkeurstemmen hebben ontvangen. In Tabel \ref{table:HoogstO} is te zien dat enkel Fred Teeven van de VVD en Jetta Klijnsma van de PVDA genoeg stemmen hebben ontvangen om boven de voorkeursdrempel uit te komen. Echter genieten deze politici een grote mate van bekendheid in Nederland en stonden zij hoog op de kandidatenlijsten. Hetgeen een goeie indicatie is voor het krijgen van voorkeurstemmen \citep{van2012tweede}. Het gemiddelde bij de hoogstgeplaatste oudere kandidaten lag op het aantal van 22.937 stemmen. Echter wordt het gemiddelde door het grote aantal voorkeurstemmen dat Jette Klijnsma ontving flink omhoog gekrikt. Het feit dat Jetta Klijnsma zo veel stemmen heeft ontvangen ligt waarschijnlijk ten grondslag aan het feit dat, zoals eerder al is aangegeven, vrouwelijke kiezers geneigd zijn om op de hoogstgeplaatste vrouwelijke kandidaat te stemmen. Het lijkt daarmee onwaarschijnlijk dat Jetta Klijnsma zo veel stemmen heeft ontvangen vanwege het feit dat zij een oudere is. Dit alles doet vermoeden dat het onwaarschijnlijk lijkt dat ouderen zich zullen committeren aan een strategie. Desalniettemin, of misschien wel des te meer, is het interessant om te onderzoek wat het theoretisch maximum is wanneer een deel van de oudere kandidaten zich committeert aan een strategie. \\

\begin{table}[H]
\centering
	\begin{footnotesize}
		\begin{tabular}{lllrr}
\toprule
{} & Geslacht &                 Partij &  Plaats   &  Aantal Stemmen  \\
 & & & op Lijst & Kandidaat\\
Kandidaat                                &          &                        &                  &                           \\
\midrule
N.P.M. (Norbert) Klein       &        M &                 50PLUS &                2 &                      3511 \\
P. (Peter) Oskam             &        M &                    CDA &               10 &                       702 \\
C.P. (Onno) van Schayck      &        M &           ChristenUnie &               17 &                       307 \\
P.H. (Paul) van Meenen       &        M &                    D66 &                7 &                      1802 \\
A. (Bram) van Ojik           &        M &             GROENLINKS &                2 &                      4639 \\
B.E.J.M. (Birgit) Verstappen &        V &  Partij voor de Dieren &                7 &                       844 \\
J. (Jetta) Klijnsma          &        V &                   PVDA &                2 &                    192190 \\
L. (Louis) Bontes            &        M &                    PVV &                5 &                       958 \\
R. (Roelof) Bisschop         &        M &                    SGP &                3 &                      2234 \\
H. (Harry) van Bommel        &        M &                     SP &                4 &                     10021 \\
F. (Fred) Teeven             &        M &                    VVD &                5 &                     35103 \\
\bottomrule
\end{tabular}

	\end{footnotesize}
			\caption{Het aantal stemmen dat de hoogstgeplaatste oudere kandidaten hebben ontvangen volgens de offci\"{e}le einduitslag.}
\label{table:HoogstO} 
\end{table}

\indent Een deelname van 100\% van de oudere kiezers aan strategie 1 zou in theorie een Tweede Kamer hebben opgeleverd waarin 89 ouderen plaats zouden nemen (zie Sectie \ref{ouderen}. Op dezelfde wijze als bij de bevolkingsgroep vrouwen is gedaan, gaan we berekenen hoeveel zetels er aan oudere kandidaten worden bedeeld wanneer bepaalde percentages van de bevolkingsgroep ouderen zich committeert aan de strategie (zie Sectie \ref{percV} voor uitleg van het bereken van het aantal stemmen). In Figuur \ref{fig:PerO} is te zien hoeveel zetels er aan oudere kandidaten zouden zijn bedeeld wanneer een bepaald percentage van de oudere kiezers zich had gecommitteerd aan strategie 1.  



\begin{figure}[H]
	\includegraphics[width=\linewidth]{percentages_van_ouderen.png}

			\caption{Grafiek met overzicht van het maximum aantal zetels wat behaald kon worden wanneer een bepaald percentage van de oudere kiezers zich committeerde aan strategie 1.}

\label{fig:PerO}
\end{figure}

In Figuur \ref{fig:PerO} is te zien dat wanneer 80\% van de oudere kiezers zich had gecommitteerd aan strategie 1, het theoretisch maximum nog altijd werd behaald. Bij 55\% van de oudere kiezers vielen er vijf oudere kandidaten af en bij slechts 35\% van de oudere kiezers zijn er nog altijd negentien ouderen meer in de Tweede Kamer dan daadwerkelijk bij de offci\"{e}le einduitslag het geval was (38 ouderen in de Tweede Kamer na de offci\"{e}le einduitslag). 

\paragraph{Keuze Strategie.}
Wat betreft het kiezen van een strategie is strategie 1 de strategie die het meeste succes opleverde bij de bevolkingsgroep ouderen. Zoals we weten, leverde strategie 1 het theoretisch maximum van 89 ouderen in de Tweede Kamer op. Echter levert strategie 2 ook een hoog rendement met 84 ouderen in de Tweede Kamer. 

\paragraph{Tegenstrategie.}
Volgens \cite{aalberts2006aantrekkelijke} vinden met name jongeren dat politici te oud zijn. Echter is het de vraag in hoeverre dit invloed zou kunnen hebben op hun partij- of kandidatenkeuze. Hoewel misschien onwaarschijnlijk is het niet ondenkbaar dat een groep opstaat en zich samen aan een strategie committeert om zodoende te zorgen dat er juist minder oudere kandidaten in de Tweede Kamer plaatsnemen. 

\subsection{Factoren bevolkingsgroep provincialen.}
\label{percP}

\paragraph{Aantal provinciale Kandidaten.} 
Bij \hyperref[S1A]{strategie 1} voor de bevolkingsgroep provincialen werd beschreven dat gezien de peiling en het bepalen van de top \textit{N} provinciale kandidaten er niet meer dan 147 provinciale kandidaten gekozen konden worden. Dit vanwege het feit dat de SP minder provinciale kandidaten op de kandidatenlijst had staan dan verwacht werd dat zij zetels zouden gaan ontvangen (zie Strategie \ref{sssec:S1P} in Sectie \ref{provincialen}). Echter de peiling correspondeerde niet 100\% met de uiteindelijke einduitslag. Op de PVDA na hadden alle partijen en hoger aantal provinciale kandidaten dan het aantal zetels dat zij a.d.h.v. de einduitslag bedeeld kregen. De PVDA had echter één zetels minder ontvangen dan het aantal provinciale kandidaten dat zij op de kandidatenlijsten hadden staan. Er hadden een totaal van 149 provincialen in de Tweede Kamer gekozen kunnen worden. Hierdoor had een Tweede Kamer bestaande uit enkel provincialen net niet mogelijk geweest. In totaal stonden er, bij de partijen die zetels hebben ontvangen, 235 oudere kandidaten op de kandidatenlijsten (zie Figuur \ref{fig:rpKandidaten} in Hoofdstuk \ref{sec:math}).

\paragraph{Committeren aan een strategie.}
In de literatuur bestaat er geen onderzoek betreffende voorkeurstemmen op provinciale kandidaten. Echter kan aan de hand van de verkiezingseinduitslag niet worden opgemaakt dat er sentimenten leven onder provincialen om beter vertegenwoordigd te worden door provinciale politici in de Tweede Kamer. In Tabel \ref{table:HoogstP} is te zien dat van de hoogstgeplaatste provincialen op de kandidatenlijsten (de lijsttrekkers buiten beschouwing gelaten) er vier provinciale kandidaten bij de offici\"{e}le verkiezingsuitslag genoeg stemmen hadden ontvangen om boven de voorkeursdrempel uit te komen. Het gemiddelde bij de hooggeplaatste provinciale kandidaten lag op 37.026 stemmen. Echter is dit waarschijnlijk het gevolg van het feit dat drie van de vier provinciale kandidaten, met meer stemmen dan de voorkeursdrempel, tevens de hoogstgeplaatste vrouwelijke kandidaten van hun partij zijn. Zoals eerder al is beschreven, kan het feit dat deze vrouwen zoveel stemmen hebben ontvangen liggen aan het feit dat zij ook de eerste vrouwelijke kandidaat op de kandidatenlijst van hun partij zijn \citep{van2012tweede}. De andere kandidaat, in de persoon van Ronald Plasterk, stond op de derde plaats op de kandidatenlijst van de PVDA. Het is goed mogelijk dat Ronald Plasterk zoveel stemmen heeft gekregen vanwege zijn hoge plaats op de kandidatenlijst enerzijds en zijn landelijke bekendheid anderzijds. Dit alles geeft onvoldoende houvast dat provincialen zich zullen committeren aan een strategie wat er voor moet zorgen dat provincialen in hogere mate vertegenwoordigd worden. \\

\begin{table}[H]
\centering
	\begin{footnotesize}
		\input{hoogste_provincialen_voorkeurstemmen_tabel3}
	\end{footnotesize}
			\caption{Het aantal stemmen dat de hoogstgeplaatste provinciale kandidaten hebben ontvangen volgens de offci\"{e}le einduitslag.}
\label{table:HoogstP} 
\end{table}

\indent Een deelname van 100\% van de provinciale kiezers aan strategie 1 zou in theorie een Tweede Kamer hebben opgeleverd waarin 138 provincialen plaats zouden nemen (zie Sectie \ref{provincialen}).  Op dezelfde wijze als bij de bevolkingsgroep vrouwen is gedaan, gaan we berekenen hoeveel zetels er aan provinciale kandidaten worden bedeeld wanneer bepaalde percentages van de bevolkingsgroep ouderen zich committeert aan de strategie (zie Sectie \ref{percV} voor uitleg van het bereken van het aantal stemmen). In Figuur \ref{fig:PerP} is te zien hoeveel zetels er aan provinciale kandidaten zouden zijn bedeeld wanneer een bepaald percentage van de provinciale kiezers zich had gecommitteerd aan strategie 4.


\begin{figure}[H]


	\includegraphics[width=\linewidth]{percentages_van_provincialen.png}

			\caption{Grafiek met overzicht van het maximum aantal zetels wat behaald kon worden wanneer een bepaald percentage van de provinciale kiezers zich committeerde aan strategie 4.}

\label{fig:PerP}
\end{figure}

In Figuur \ref{fig:PerP} is te zien dat wanneer slechts 50\% van de provinciale kiezers zich had gecommitteerd aan strategie 4, er maar vijf provinciale kandidaten afvielen. Bij 45\% van de provinciale kiezers zouden er nog eens acht provinciale kandidaten afgevallen zijn, bij 40\% van de provinciale kiezers nog eens twintig en bij 35\% zouden er niet meer provinciale kandidaten in de Tweede Kamer komen dan daadwerkelijk al bij de offci\"{e}le verkiezingsuitslag al het geval was (75 provincialen in de Tweede Kamer na offci\"{e}le einduitslag).

\paragraph{Keuze Strategie.}
Wat betreft het kiezen van een strategie is strategie 1 niet de strategie die het meeste succes opleverde bij de bevolkingsgroep provincialen. Zoals aangetoond bij de uitvoering van \hyperref[S4P]{strategie 4} in Bijlage \ref{provincialen}, leverde strategie 4 het theoretisch maximum van 142 provincialen in de Tweede Kamer. Hierbij diende de \textit{N} te worden uitgebreid met een \textit{extra percentage=20}. Echter levert ook strategie 1 een hoog rendement op met 138 provincialen in de Tweede Kamer. 

\paragraph{(Tegen)strategie.}
Vanuit de literatuur is er geen directe indicatie te vinden die erop wijst dat er daadwerkelijke anti-provincialen sentimenten bestaan bij Randstedelingen. De literatuur beschrijft wel een gebrek aan een Randstad-identiteit bij inwoners van de Randstad \citep{meijers2001deltametropool,terlouw2010randstad}. Volgens \cite{terlouw2010randstad} echter is het goed mogelijk dat er wel een Randstedelijke identiteit kan ontstaat wanneer er besloten wordt om één grote Randstad-provincie te gaan vormen. Hoewel het vanuit het huidige tijdsbeeld onwaarschijnlijk lijkt dat de Randstedelingen gezamenlijk een (tegen)strategie zullen uitvoeren om te voorkomen dat de provincialen in hogere mate vertegenwoordigd zullen worden in de Tweede Kamer, is het niet onmogelijk dat de Randstedelingen (of een andere groep zoals bijvoorbeeld een groep bestaande uit Randstedelingen én provincialen) op termijn een (tegen)strategie zullen uitvoeren. 


