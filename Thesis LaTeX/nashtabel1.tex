



\iffalse
\begin{tabular}{|r|r|r|r|r|r|}
\hline
V   & M   & V   & M   & V   & M   \\ \hline
6   & 4  & NaN & NaN & NaN & NaN \\ \hline
7   & 3   & 5   & 5 \tikzmark{f}   & NaN & NaN \\ \hline
NaN & NaN & 8   & \tikzmark{d}{2} \tikzmark{e}  & \tikzmark{c}{4}   & \tikzmark{b}{6}   \\ \hline
NaN & NaN & NaN & NaN & 10  & \tikzmark{a}{0}   \\ \hline
\end{tabular}
\begin{tikzpicture}[overlay, remember picture, shorten >=.5pt, shorten <=.5pt]

   \draw [->] ({pic cs:a}) [line width=0.35mm, yshift=-1] to ({pic cs:b});
    \draw [->] ({pic cs:c}) [line width=0.35mm, yshift=-1] to ({pic cs:d});
    \draw [->] ({pic cs:e}) [line width=0.35mm, yshift=-1] to ({pic cs:f});
\end{tikzpicture}
\fi


%\left\Downarrow% Use `\left.` if don't want arrow on this side.
\begin{tabular}{llrr}
\toprule
 {}                                       & {}            &   Aantal Zetels & Aantal  Zetels \\
 Strategie (op volgorde van aanpassing)                    &                          {}   & Vrouwen       & Mannen \\
\midrule
Beiden bevolkingsgroepen dezelfde strategie & \tikzmark{a}{} & 10 &  0\\
Mannelijke kiezers passen strategie aan   &   {}          & 4  &  6 \\
Vrouwelijke kiezers passen strategie aan  &    {}          & 8  &  2\\
Mannelijke kiezers passen strategie aan   &    {}          & 5  &  5 \\
Vrouwelijke kiezers passen strategie aan  &    {}          & 7  &  3 \\
Mannelijke kiezers passen strategie aan   & \tikzmark{b}{} & 6  &  4 \\
\bottomrule
\end{tabular}
%\right.


\begin{tikzpicture}[overlay, remember picture, shorten >=.5pt, shorten <=.5pt]

   \draw [->] ({pic cs:a}) [line width=0.4mm, yshift=2, ] to ({pic cs:b});

\end{tikzpicture}


