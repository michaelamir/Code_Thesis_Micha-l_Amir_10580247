\newpage
\section{Tegenbeweging en de Zoektoch naar het Nash Equilibrium.}
\label{h7}

\subsection*{Deelvraag 4: Wat kan er gebeuren wanneer een bevolkingsgroep zich committeert aan een strategie en een andere bevolkingsgroep zich committeert aan een tegenstrategie?}


In Hoofdstuk \ref{sec:eva} hebben we beschreven en berekend hoe bevolkingsgroepen de regel van de voorkeursdrempel in hun voordeel kunnen gebruiken wanneer zij zich gezamenlijk committeren aan een strategie. In dit hoofdstuk gaan we onderzoek wat er kan gebeuren wanneer een andere bevolkingsgroep (met tegengestelde eigenschappen) zich committeert aan een tegenstrategie om zodoende een voordeel te kunnen behalen. Hierbij zullen we een denkbeeldige partij gebruiken die we de \textit{Nash-partij} noemen. In het voorbeeld zijn er twee bevolkingsgroepen: mannelijke kiezers en vrouwelijke kiezers. Interessant is hierbij te onderzoeken of er een \textit{Nash Equilibrium} bereikt kan worden. Een Nash Equilibrium werd voor het eerst beschreven door de Amerikaanse wiskundige en economoon John Forbes Nash Jr. \cite{nash1950equilibrium} en kan als volgt worden omschreven: een toestand van een systeem met meerdere participanten waarbij geen van de participanten een voordeel kan behalen d.m.v. het veranderen van strategie zolang alle andere participanten oververanderd blijven in hun strategie \citep{christiansen2016neuroeconomics, nashprinceton}. Oftewel is er een toestand te bereiken wanneer het uitvoeren van strategie\"{e}n door (tegengestelde) bevolkingsgroepen geen extra voordeel meer oplevert wanneer een groep zijn/haar strategie aanpast maar de andere groep onveranderd blijft. 

\subsubsection*{Aannames.} \label{aannamesNash}
In deze sectie zijn er allereerst een aantal aannames benodigd voordat we kunnen gaan onderzoeken wat er gebeurt wanneer twee (tegengestelde) bevolkingsgroepen zich committeren aan een strategie. Ten behoeve van de leesbaarheid en begrijpelijkheid zullen we deze aannames doen om zodoende te gaan onderzoeken of er een Nash Equilibrium bereikt kan worden. Deze aannames zijn:

\begin{itemize}
	\item
In de zoektocht naar het Nash Equilibrium kijken we naar één partij. Deze partij noemen we, als eerbetoon aan John Forbes Nash Jr., de Nash-partij. Voor de Nash-partij geldt:
		\begin{itemize}
			\item
			De partij verwacht het aantal van tien zetels te gaan ontvangen bij de verkiezingen. Het aantal zetels noemen we \textit{Z}. Derhalve geldt voor de Nash-partij $Z=10$
			\item
			Het aantal vrouwelijke en mannelijke kandidaten op de kandidatenlijst van de Nash-partij is gelijk verdeeld met tien om tien. Er staan dus in totaal twintig kandidaten op de kandidatenlijst. 
			\item
			De vrouw-man stemmenverdeling is 60\% om 40\%. Oftewel 60\% van de stemmen op de Nash-partij worden uitgebracht door vrouwelijke kiezers en 40\% van de stemmen op de Nash-partij worden uitgebracht door mannelijke kiezers. Hierbij nemen we \textit{P} als de kans dat een stem op de Nash-partij door een vrouwelijke kiezer wordt uitgebracht en ($1-P$) als de kans dat een stem op de Nash-partij door een mannelijke kiezer wordt uitgebracht. In dit geval geldt dus $P=0,6$ en ($1-P)=0,4$.
			\item
			Zowel 100\% van de vrouwelijke kiezers als van de mannelijke kiezers die hun stem uitbrengen op de Nash-partij doen mee aan de strategie.
		\end{itemize}
	\item
We stellen een hypothetische verkiezing op. Deze verkiezingen zijn in opzet hetzelfde als de Tweede Kamerverkiezingen in Nederland. Bij deze verkiezingen geldt:
		\begin{itemize}
			\item 
		De kiesdeler is zestig stemmen. Oftewel één zetels staat gelijk aan zestig stemmen.
		\item
		De voorkeursdrempel is dertig stemmen. Een kandidaat heeft dus minstens dertig stemmen nodig om met voorkeur gekozen te worden.
		\end{itemize}
\end{itemize}

\begin{theorem} "De toestand van een Nash Equilibrium wordt bereikt wanneer de verdeling van stemmen \textit{P/(1-P)} op een partij gelijk is aan de verdeling van het aantal zetels (\textit{Z}) voor vrouwelijke en mannelijke kandidaten waarbij geldt dat de zetelverdeling \textit{P*Z/(1-P)*Z} is. Dit is tevens het enige Nash Equilibrium." \\
\end{theorem}

Om de bovenstaande stelling te bevestigen of te ontkrachten zullen we hieronder een aantal voorbeelden schetsen. Zoals hierboven is bepaald, kan de Nash-partij tien zetels (\textit{Z=10}) gaan verwachten. Dit komt neer op het aantal van ($10*60$ = ) 600 stemmen. Van de zeshonderd stemmen op de Nash-partij is de kans $P=0,6$ dat een stem afkomstig van een vrouwelijke kiezer. Dit komt neer op het aantal van ($0,6*600$ = ) 360 stemmen van vrouwelijke kiezers. Het aantal stemmen dat afkomstig is van mannelijke kiezers komt daarmee uit op ($600*(1-P)=600-360$ = ) 240 stemmen.

Het eerste scenario dat we bespreken gaat als volgt: zowel de vrouwelijke als de mannelijke kiezers hanteren dezelfde strategie. Bij deze strategie worden de stemmen willekeurig uitgebracht op één van de eerste tien kandidaten op de kandidatenlijst. Zodoende krijgen de eerste tien vrouwelijke kandidaten op de kandidatenlijst van de Nash-partij het aantal van ($360\div10$ = ) 36 stemmen per kandidaat. De mannelijke kandidaten echter krijgen slechts het aantal van($240\div10$ = ) 24 stemmen per kandidaat. Vanwege het feit dat de voorkeursdrempel op dertig stemmen is vastgesteld, worden de tien zetels die de Nash-partij ontvangt bedeeld aan tien vrouwelijke kandidaten en is er dus geen enkele mannelijke kandidaat verkozen.

Er vanuit gaande dat de beiden groepen op de hoogte zijn wat elkaars strategie, kunnen de mannelijke kiezers hun strategie aanpassen om te voorkomen dat de tien zetels van de Nash-partij aan enkel vrouwelijke kandidaten worden bedeeld. Zo kunnen de mannelijke kiezers de stemmen over de eerste zes kandidaten verdelen in plaats van over de eerste tien. Op deze wijze komen de eerste zes mannelijke kandidaten op het aantal van ($240\div6$ = ) 40 stemmen per kandidaat. Wanneer de vrouwelijke kiezers hun strategie niet aanpassen en derhalve de eerste tien vrouwelijke kandidaten het aantal van 36 stemmen per kandidaat bezorgen, worden er zes zetels aan de mannelijke kandidaten bedeeld en vier zetels aan de vrouwelijke kandidaten bedeeld. 

Wanneer vrouwelijke kiezers er echter wel voor kiezen om hun strategie aan te passen om daarmee een voordeel te behalen, kunnen zij hun stemmen verdelen over de eerste acht vrouwelijke kandidaten van de Nash-partij in plaats van de eerste tien. Op deze wijze komen de eerste acht vrouwelijke kandidaten op het aantal van ($360\div8$ = ) 45 stemmen per kandidaat. Wanneer de mannelijke kiezers hun strategie niet verder aanpassen en derhalve de eerste zes mannelijke kandidaten het aantal van 40 stemmen per kandidaat bezorgen, worden er acht zetels aan vrouwelijke kandidaten bedeeld en twee aan mannelijke kandidaten. 

In Tabel \ref{table:nashtab} is de zien wat er gebeurt wanneer deze ontwikkeling van het aanpassen van de strategie zich doorzet. Daarbij moet genoteerd worden dat elke keer dat een bevolkingsgroep zijn of haar strategie aanpast het zo is dat de andere groep de laatste gekozen strategie hanteert. 


\begin{table}[H]
\centering

		



\iffalse
\begin{tabular}{|r|r|r|r|r|r|}
\hline
V   & M   & V   & M   & V   & M   \\ \hline
6   & 4  & NaN & NaN & NaN & NaN \\ \hline
7   & 3   & 5   & 5 \tikzmark{f}   & NaN & NaN \\ \hline
NaN & NaN & 8   & \tikzmark{d}{2} \tikzmark{e}  & \tikzmark{c}{4}   & \tikzmark{b}{6}   \\ \hline
NaN & NaN & NaN & NaN & 10  & \tikzmark{a}{0}   \\ \hline
\end{tabular}
\begin{tikzpicture}[overlay, remember picture, shorten >=.5pt, shorten <=.5pt]

   \draw [->] ({pic cs:a}) [line width=0.35mm, yshift=-1] to ({pic cs:b});
    \draw [->] ({pic cs:c}) [line width=0.35mm, yshift=-1] to ({pic cs:d});
    \draw [->] ({pic cs:e}) [line width=0.35mm, yshift=-1] to ({pic cs:f});
\end{tikzpicture}
\fi


%\left\Downarrow% Use `\left.` if don't want arrow on this side.
\begin{tabular}{llrr}
\toprule
 {}                                       & {}            &   Aantal Zetels & Aantal  Zetels \\
 Strategie (op volgorde van aanpassing)                    &                          {}   & Vrouwen       & Mannen \\
\midrule
Beiden bevolkingsgroepen dezelfde strategie & \tikzmark{a}{} & 10 &  0\\
Mannelijke kiezers passen strategie aan   &   {}          & 4  &  6 \\
Vrouwelijke kiezers passen strategie aan  &    {}          & 8  &  2\\
Mannelijke kiezers passen strategie aan   &    {}          & 5  &  5 \\
Vrouwelijke kiezers passen strategie aan  &    {}          & 7  &  3 \\
Mannelijke kiezers passen strategie aan   & \tikzmark{b}{} & 6  &  4 \\
\bottomrule
\end{tabular}
%\right.


\begin{tikzpicture}[overlay, remember picture, shorten >=.5pt, shorten <=.5pt]

   \draw [->] ({pic cs:a}) [line width=0.4mm, yshift=2, ] to ({pic cs:b});

\end{tikzpicture}




			\caption{De verdeling van de zetels wanneer één van de twee bevolkingsgroepen de strategie aanpast terwijl de andere groep de laatste gekozen strategie hanteert. Het Nash Equilibrium wordt bereikt bij zes zetels voor de vrouwen en vier zetels voor de mannen.}
\label{table:nashtab} 
\end{table}

Zoals te zien in Tabel \ref{table:nashtab} hierboven toont de onderste rij in de tabel het punt waarop de vrouwelijke kandidaten het aantal van zes zetels hebben ontvangen en de mannelijk kandidaten het aantal van vier zetels hebben ontvangen. Hiermee is volgens de eerdere genoemde stelling het Nash Equilibrium bereikt. De verhouding van de verdeling van het aantal zetels is nu namelijk (\textit{P*Z/(1-P)*Z} = $0,6*10/(1-0,6)*10$ = ) 6/4. Echter is het de vraag of daadwerkelijk het Nash Equilibrium is bereikt. Om te bewijzen dat het Nash Equilibrium daadwerkelijk is bereikt, en dat dit teven het enige Nash Equilibrium is, gaan we nogmaals de strategie van de vrouwelijke en de mannelijk kiezers aanpassen vanuit de laatste rij in Tabel \ref{table:nashtab}. 

In de laatste rij in de tabel hebben de mannelijke kiezers hun stemmen verdeeld over de eerste vier mannelijke op de kandidatenlijst. Op de wijze komen de eerste vier mannelijke kandidaten op het aantal van ($240\div4$ = ) 60 stemmen per kandidaat. De vrouwelijke kiezers echter bleven constant in hun laatst gekozen strategie. Daarbij verdeelden zij de stemmen over zeven vrouwelijke kandidaten. Zodoende kwamen de eerste zeven vrouwelijke kandidaten uit op het aantal van ($360\div7$ = ) 51 stemmen per kandidaat. De eerste vier mannelijke kandidaten hebben dus ($60-51$ = ) 9 stemmen meer per kandidaat. Hierdoor worden de eerste vier zetels bedeeld aan de eerste vier mannelijke kandidaten op de kandidatenlijst en de andere zes zetels worden bedeeld aan de eerste zes vrouwelijke kandidaten op de kandidatenlijst. De vrouwelijke kandidaten hebben immers minder stemmen per kandidaat. 

Mannelijke kiezers kunnen nu ervoor kiezen om hun strategie nogmaals aan te passen en de stemmen te verdelen over de eerste drie mannelijke kandidaten op de kandidatenlijst. Op deze wijze komen de eerste drie mannelijke kandidaten op het aantal van ($240\div3$ = ) 80 stemmen per kandidaat. Echter hebben de overige mannelijke kandidaten geen enkele stem ontvangen. Vanwege het feit dat de eerste zeven vrouwelijke kandidaten 51 stemmen per kandidaten hebben ontvangen, gaan de mannelijke kandidaten er dus op achteruit. Bij de voorlaatste strategie hadden zij immers vier zetels bedeeld gekregen. Bij de laatste strategie zijn dat er nog maar drie zetels.

In het volgende voorbeeld passen de vrouwelijke kiezers hun strategie aan. Daarbij hebben de mannelijke kiezers hun strategie niet aangepast zoals in het laatste voorbeeld hierboven. De mannelijke kiezers verdelen hun stemmen dus over de eerste vier mannelijke kandidaten op de kandidatenlijst. In plaats van dat de vrouwelijke kiezer nu hun stemmen gaan verdelen over de eerste zeven, zoals in hun laatst toegepaste strategie, verdelen zij de stemmen over de eerste zes vrouwelijke kandidaten op de kandidatenlijst. Op deze wijze komen de vrouwelijke kandidaten uit op het aantal van ($360\div6$ = ) 60 stemmen per kandidaat. Zowel de eerste zes vrouwelijke alsmede eerste zes mannelijke kandidaten hebben nu het aantal van 60 stemmen per kandidaat. Daarmee zijn er zes vrouwelijke kandidaten gekozen en vier mannelijke kandidaten. De vrouwelijke kiezers kunnen nu ervoor kiezen om hun strategie nogmaals aan te passen en de stemmen te verdelen over de eerste vijf vrouwelijke kandidaten op de kandidatenlijst. Op deze wijze komen de eerste vijf vrouwelijke kandidaten op het aantal van ($360\div5$ = ) 72 stemmen per kandidaat. Echter hebben de overige vrouwelijke kandidaten geen enkele stem ontvangen. Zodoende worden de eerste vijf zetels van de Nash-partij bedeeld aan de eerste vijf vrouwelijke kandidaten op de kandidatenlijst. Vanwege het feit dat de eerste vier mannelijke kandidaten 60 stemmen per kandidaat hebben ontvangen, worden er vier zetels bedeeld aan mannelijke kandidaten. Er blijft zodoende nog één zetel over. Deze zetel kan voor een mannelijke of vrouwelijke kandidaat zijn. Welke kandidaat deze zetels bedeeld krijgt ligt aan het gegeven of een mannelijke of een vrouwelijke kandidaat op de eerstvolgende plaats op de kandidatenlijst staat die recht geeft op de zetel. Echter in het geval de zetel wordt bedeeld aan een vrouwelijke kandidaat, gaan de vrouwelijke kiezers er niet op vooruit. Zij hadden immers bij de laatste strategie al het aantal van zes zetels behaald en bij deze strategie kunnen zij ook maximaal zes zetels behalen. 

Zodoende is het Nash Equilibrium bereikt bij de zetelverdeling van zes zetels voor vrouwelijke kandidaten en vier zetels voor mannelijke kandidaten. Geen van de bevolkingsgroepen kan erop vooruit wanneer zij hun strategie aanpassen terwijl de andere bevolkingsgroep de strategie niet aanpast. Hierbij moet genoteerd worden dat, hoewel in het voorbeeld twee bevolkingsgroepen zijn gebruikt, het Nash Equilibrium op dezelfde manier ook kan worden bereikt wanneer er meer dan twee bevolkingsgroepen zich committeren aan een strategie. Hierbij geldt dan: $\sum_{b} P{b} * Z = Z$. Oftewel, de som van de aandeel stemmen van de bevolkingsgroepen vermenigvuldigd met het aantal zetels is gelijk aan het zetels.




 
