\section{Gerelateerd Werk}
\label{sec:rel}
Zoals in de introductie al is genoemd, hanteert Nederland een evenredig kiesstelsel en een dergelijk kiesstelsel positief bij aan de vertegenwoordiging van vrouwen en minderheden. In 2005 stelde toenmalig minister De Graaf van Bestuurlijke Vernieuwing voor om het kiesstelsel om te gooien en een tweestemmenstelsel te combineren met een stelsel van meervoudige kiesdistricten. Hierbij zouden de kiezers zowel op een lokale kandidaat stemmen alsmede een landelijke kandidaat. In het onderzoek van \cite{norris2006impact} is berekend dat het voorstel van minister De Graaf een negatieve invloed zou hebben op het aantal vrouwelijke kamerleden. Gelukkig voor de vrouwen in Nederland is het voorstel van de minister verworpen. Desalniettemin is Nederland een land wat relatief weinig maatregelen heeft genomen als het gaat om het adequater vertegenwoordigen van vrouwen en tevens ook van minderheden. Daarentegen zijn met name in de laatste twintig á dertig jaar er wereldwijd steeds meer maatregelen in het leven geroepen die ervoor hebben moeten zorgen dat vrouwen en minderheden adequater vertegenwoordigd worden in volksvertegenwoordigingen \citep{norris2007opening}. Een van de maatregelen die in het leven is geroepen in een quotum. De quota hebben wereldwijd hoofdzakelijk drie soorten vormen. Hier gaan we in dit hoofdstuk in de eerstvolgende sectie dieper op in.

Naast dat we dieper ingaan op de maatregelen die genomen kunnen worden om vrouwen en minderheden adequater te vertegenwoordigen, zullen we ook kijken naar IT-toepassingen en het effect daarvan op het vergroten van de betrokkenheid en participatie van stemgerechtigden en andere burgers. In Hoofdstuk \ref{h6} zullen we een tweetal IT-toepassingen (en een derde die een mengvorm is van de eerste twee) voorstellen die moeten bijdragen aan de in dit onderzoek ontwikkelde stratie\"{e}n zoals wordt beschreven in Hoofdstuk \ref{sec:eva}. In het huidige hoofdstuk zullen derhalve ook dieper ingaan op enkele vormen van IT-toepassingen die al zijn ontwikkeld voor de Nederlandse Tweede Kamerverkiezingen. 

\subsection{Quota.} 
Wereldwijd zijn er drie soorten quota te onderscheiden die eraan bijdragen dat vrouwen en minderheden adequater vertegenwoordigd worden in volksvertegenwoordigingen. Deze quota zijn:  gereserveerde zetels, wettelijk vastgelegde quota en vrijwillige quota \citep{norris2007opening}. 

Gereserveerde zetels zijn zetels die specifiek voor leden van een bepaalde bevolkingsgroep zijn vastgesteld. De partijen die deze zetels ontvangen moeten deze zetels opvullen met een kandidaat afkomstig uit de bevolkingsgroep waarvoor deze bedoeld is. In sommige landen is het zo dat wanneer dit niet gebeurt de zetel wordt vergeven aan een andere partij die wel aan de wet regel kan voldoen. Het quotum-systeem van de gereserveerde zetel wordt met name toegepast in jonge (transitionele) democrati\"{e}n. Bij een wettelijk vastgelegde quotum is het zo dat er in de wet staat hoeveel procent van de verkiezingskandidaten van een partij vanuit een bepaalde bevolkingsgroep afkomstig moet zijn. In Europa hanteren bijvoorbeeld Belgi\"{e}, Frankrijk en Spanje een dergelijk quotum
 \citep{council2012positive}. Bij een vrijwillige quotum tot slot, is de partij vrij om zelf te bepalen of zij een quotum invoeren of niet. Er is niets in de wet vastgelegd dat partijen een quotum moeten hanteren. Nederland is een land waar een vrijwillig quotum bestaat. De meeste partijen hadden een redelijk aantal vrouwelijke kandidaten op de kandidatenlijsten hadden staan bij de verkiezingen van 2012. Echter had de SGP geen enkele vrouwelijke kandidaat op de kandidatenlijst staan. De reden hiervoor ligt ten grondslag aan religieuze overtuigingen die er binnen de partij heersen. Echter moet de SGP van het Europese Hof van de Rechten van de Mens bij de eerstvolgende verkiezingen ook vrouwelijke kandidaten opnemen op de kandidatenlijst \citep{Nudef30:online}. Wat betreft de bevolkingsgroep allochtonen waren er zelfs twee partijen die geen allochtone kandidaten op de kandidatenlijst hadden staan. Dit waren de Partij voor de Dieren en (nogmaals) de SGP. Andere West-Europese landen die een vrijwillig quotum-systeem hanteren zijn Duitsland, Zwitserland en Groot Brittannië \citep{Quota47:online}. Een onderzoek van \cite{ryan2010politics} wijst uit dat in Groot Brittannië de partijen in veel gevallen een eigen quotum hanteren als het gaat om het aantal vrouwen op de kandidatenlijsten voor verschillende verkiezingen. Tevens blijkt uit dit onderzoek dat de vrouwelijke kandidaten vooral door deze partijen naar voren worden geschoven in districten (Groot Brittannië hanteert een districtenstelsel) waar de partij een kleine kans maakt om te winnen. 

Zoals hierboven al genoemd is ook Nederland een land waar een vrijwillig quotum-systeem wordt gehanteerd.  Doch hanteert geen enkele partij een quotum als het gaat om het aantal kandidaten afkomstig uit één van de in dit onderzoek beschreven bevolkingsgroepen. Sterker nog, de SGP wil geen vrouwelijke kandidaten op de kandidatenlijst. Het ontbreken van een quotum bij de Nederlandse politieke partijen geeft des te meer redenen voor leden van de vier bevolkingsgroepen om zich te committeren aan een strategie waardoor zij adequater vertegenwoordigd kunnen worden in de Tweede Kamer.

\subsection{IT-toepassingen bij verkiezingen.}
Verder in dit onderzoek stellen we een tweetal IT-toepassingen voor die leden van de bevolkingsgroep kunnen helpen bij het uitvoeren van een strategie (zie Hoofdstuk \ref{h6}). In Nederland zijn er de laatste jaren een aantal IT-toepassingen ontwikkeld om die dienen ter ondersteuning van de kiezers. Zo wordt de \textit{StemWijzer} al sinds de Tweede Kamerverkiezingen van 1998 in een online vorm gebruikt door vele kiezers \citep{kleinnijenhuis2007nederland}. Bij de StemWijzer worden de kiezers enkele stellingen voorgelegd. De kiezer kan aangeven in hoeverre zij/hij het met de stelling eens is. Aan de hand van de mening van de kiezer wordt vervolgens een advies gegeven over welke partij het beste bij de opinie van de kiezer past. Een vergelijkbare toepassing is het \textit{Kieskompas}. De werking van het Kieskompas is vrijwel hetzelfde als die van de StemWijzer. Echter stellen \cite{kleinnijenhuis2007nederland} in hun onderzoek dat bij de Tweede Kamerverkiezingen van 2006 de SP een voordeel had tegenover andere partijen. De VVD en de PVDA waren in het nadeel in vergelijking met andere partijen. Bij het Kieskompas waren vooral de kleine rechtse partijen in het voordeel en de VVD ook hierbij in het nadeel in vergelijking met andere partijen. Tevens stellen zij niet alle factoren in beschouwing worden genomen in het geven van een advies door zowel de StemWijzer als het Kieskompas. Zodoende komt het voor dat sommige kiezers een incorrect stemadvies krijgen. Uit onderzoek van \cite{garzia2012voting} blijkt dat 38\% van de stemgerechtigden de StemWijzer of het Kieskompas hebben gebruikt om tot hun stem te komen. Ook blijkt uit hun onderzoek dat er soortgelijke toepassingen zijn in andere Europese landen zoals bijvoorbeeld Duitsland, Belgi\"{e} en Groot Brittani\"{e}. Naast de StemWijzer en het Kieskompas bestaan er in Nederland ook soortgelijke online toepassingen voor specifieke bevolkingsgroepen. Zo is er een stemwijzer voor jongeren \citep{Jonge36:online}, was er bij de Tweede Kamerverkiezingen van 2012 een stemwijzer voor ouderen \citep{Stemw79:online,Stemw68:online} en is er zelfs een stemwijzer voor cannabisgebruikers \citep{Canna56:online}.

Naast de toepassingen die dienen als hulpmiddel bij het maken van een partijkeuze is er ook een mobiele applicatie die ervoor moet zorgen dat er meer jongeren naar de stembus gaan \citep{Verki80:online}. Deze mobiele applicatie, de \textit{Verkiezingsapp} genoemd, poogt jongeren op een moderne manier enthousiast te maken voor verkiezingen. In de mobiele applicaties komen thema's naar voren waar jongeren mee te maken krijgen. Tevens zijn de gebruikers in staat hun eigen meningen en bevindingen te delen met andere jongeren via \textit{social media}. Enig onderzoek van wetenschappelijke aard ontbreekt echter als het gaat om het effect van de Verkiezingsapp. 

In de literatuur wordt er een positief effect beschreven als het gaat om IT-toepassingen en het vergroten van betrokkenheid en participatie van stemgerechtigden en andere burgers tijdens verkiezing en tevens voor het democratische politieke proces als geheel \citep{drezner2008power,doostdar2004vulgar,gimmler2001deliberative}. 










\iffalse
Deze sectie bestaat uit een aantal "blokken", waarin je per blok de relevante literatuur beschrijft. 

Neem alleen literatuur op die van belang is voor jouw onderzoeksvraag en deelvragen.

Typisch heb je 1 blok voor je hoofdvraag en per deelvraag \textbf{RQi} een blok. 


\subsection{RQ1}

\subsection{RQ2}
\fi