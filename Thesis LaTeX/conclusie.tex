\newpage
\section{Conclusie}
In de Nederlandse Tweede Kamer zijn de bevolkingsgroepen vrouwen, allochtonen, ouderen (personen met een leeftijd van vijftig jaar en ouder) en provincialen (personen die niet in de Randstad wonen) ondervertegenwoordigd. Het aandeel zetels in de Tweede Kamer dat deze groepen bij de laatste Tweede Kamerverkiezingen (2012) hebben ontvangen is geen directe afspiegeling van het aandeel stemgerechtigden afkomstig uit deze bevolkingsgroepen. Vanwege dit feit en vanwege het ontbreken van een enige vorm van quotum bij de Nederlandse politieke partijen zijn er in dit onderzoek een vijfta strate\"{e}n ontwikkeld die de bevolkingsgroepen zouden moeten kunnen helpen om in hogere mate vertegenwoordigd te worden dan bij de einduitslag van de Tweede Kamerverkiezingen in 2012 het geval was. 

Er is aangetoond dat de ontwikkelde strate\"{e}n goed werkten wanneer deze zouden zijn uitgevoerd ten tijde van de Tweede Kamerverkiezingen van 2012. Strategie 1 (top \textit{N} a.d.h.v. de peiling) zou voor alle bevolkingsgroepen een hoog rendement hebben behaald. Bij de bevolkingsgroepen Vrouwen, allochtonen en ouderen zou deze strategie zelfs het hoogste rendement hebben opgeleverd. Strategie 2 (willekeurig stemmen op één van alle kandidaten) zou eveneens voor alle bevolkingsgroepen een hoog rendement hebben opgeleverd. Bij de bevolkingsgroep allochtonen zou het zelfs niets hebben uitgemaakt of strategie 1 of strategie 2 zou zijn uitgevoerd. Beiden zouden hetzelfde aantal zetels voor deze bevolkingsgroep hebben opgeleverd. Tevens is Strategie 2 voor de bevolkingsgroep vrouwen het makkelijkst om uit te voeren. Zij hebben daar geen enkel hulpmiddel bij nodig vanwege het feit dat het geslacht van de kandidaat altijd op het stemformulier vermeld staat. Strategie 4 (top \textit{N+extra percenta}) zou voor de bevolkingsgroep provincialen het hoogste rendement hebben opgeleverd. Strategie 3.1 en strategie 3.2 zouden voor alle bevolkingsgroepen geen hoog rendement hebben opgeleverd en zodoende is het af te raden deze twee strategie\"{e}n ten uitvoer te brengen.

Of het maximum aantal mogelijk kandidaten behaald kan worden ligt een aantal factoren. Om mee te beginnen moeten de partijen waarom een kiezer stemt kandidaten hebben afkomstig uit de bevolkingsgroep van deze kiezer. Ten tijde van de Tweede Kamerverkiezingen van 2012 had de SGP geen vrouwelijke kandidaten en geen allochtone kandidaten op de kandidatenlijst staan. De Partij voor de Dieren had geen allochtonen op de kandidatenlijst staan. Ten tweede dient er vanuit de bevolkingsgroep animo te zijn om zich als collectief (of als substantieel aandeel) achter een strategie te scharen. Afgaande op de bevindingen in Hoofdstuk \ref{h5}, lijkt er bij de bevolkingsgroep vrouwen de meeste animo te zijn voor het uitvoeren van een strategie om zodoende adequater vertegenwoordigd te worden in de Tweede Kamer. Echter, meer onderzoek is nodig bij deze bevolkingsgroep in Nederland om te weten of dit daadwerkelijk zo is. Voor alle groepen geldt dat strategie 1 en strategie 2 bij een deelname van 50\% van de bevolkingsgroep al een substantieel hoger aantal zetels voor de bevolkingsgroepen zou zijn gehaald. Hierbij wordt dan wel aangenomen dat er telkens één bevolkingsgroep een strategie zou hanteren. Ten derde kan het behalen van het maximum uiteraard be\"{i}nvloed worden door de keuze van de strategie. De strategi\"{e} hebben allemaal andere regels voor uitvoering en dit brengt een zeker complexiteit met zich mee. Enkel strategie 2 voor de bevolkingsgroep vrouwen is, zoals hierboven al vermeld, gemakkelijk uit te voeren voor deze bevolkingsgroep.  Ter ondersteuning van het uitvoeren van de strate\"{e}n, hebben we in Hoofdstuk \ref{h6} een tweetal IT-toepassingen voorgesteld. De eerste toepassing toont de kiezer alle uit de eigen bevolkingsgroep afkomstige kandidaten van de partij waar zij/hij op wil gaan stemmen. De kiezer kan dan zelf willekeurig hier een kandidaat uit kiezen. Echter blijkt dat mensen niet adequaat zijn in het willekeurig kiezen. Vandaar dat een een tweede toepassing is voorgesteld. Bij de tweede toepassing geeft de kiezer aan welke partij zij/hij wil gaan stemmen en waarna een willekeurige kandidaat van deze partij wordt toegewezen aan de kiezer.  De tweetal toepassingen dienen slechts als voorstellen en meer onderzoek is dan ook nodig om te bepalen welke functionaliteiten een toepassing daadwerkelijk zou moeten bezitten. De laatste factor die van invloed kan zijn op het behalen van het maximum is een andere bevolkingsgroep (of bevolkingsgroepen) die een (tegen)strategie hanteert. In dit onderzoek is aangetoond dat een Nash Equilibrium kan worden bereikt wanneer twee of meerde groepen zich committeren aan een strategie. Bij \textit{P=percentage stemmen afkomstig uit een bevolkingsgroep in verhouding tot totaal aantal stemmen} en \textit{Z=aantal zetels} geldt: \textit{P*Z/(1-P)*Z} voor een Nash Equilibrium.

Om te voorkomen dat één of meerdere bevolkingsgroepen de voorkeursregel dusdanig uitbuiten dat zij oververtegenwoordigd zijn, hebben we aangetoond dat uitbuiting niet meer mogelijk is bij het verhogen van de voorkeursdrempel van een kwart van het aantal stemmen van de kiesdeler naar het totaal aantal stemmen van de kiesdeler. Bevolkingsgroepen kunnen dan nog altijd een strategie uitvoeren, echter kunnen zij in geen geval beter vertegenwoordigd worden dan de directe afspiegeling van het aandeel stemgerechtigden. Zodoende luidt de eerste aanbeveling van dit onderzoek: om uitbuiting van de regel van de voorkeursdrempel door één of meerdere bevolkingsgroepen te voorkomen, verhoog de voorkeursdrempel naar hetzelfde aantal stemmen als de kiesdeler. 
Zoals hierboven al is vermeld, lijkt er bij de bevolkingsgroep vrouwen waarschijnlijk het meeste animo te bestaan voor het uitvoeren van een strategie. Aangezien het feit dat het geslacht van de kandidaten al op het stemformulier te vinden is en in het geval het eerste advies zal worden opgevolgd doen we ook een tweede advies. Dit advies luidt: een eigen hokje voor vrouwelijke kandidaten en de mannelijk kandidaten. Het is daarbij de bedoeling dat de alle stemmen waarbij het hokje 'vrouwelijke kandidaat' is ingevuld worden verdeeld over alle vrouwen op de kandidatenlijst van de partij. Dit geldt uiteraard ook voor mannelijke kandidaten en hun eigen hokje. In het geval het eerst advies is opgevolgd en alle vrouwelijke kiezers vinken dit hokje aan, wordt het aandeel zetels wat vergeven wordt aan vrouwelijke kandidaten een direct afspiegeling van het aandeel stemmen van vrouwen in verhouding tot het totaal aantal stemmen.

\subsection*{Beperkingen onderzoek.}
Meer onderzoek is nodig naar wat er echt speelt binnen leden van de in dit onderzoek beschreven bevolkingsgroepen. De bevolkingsgroepen vrouwen en daarna allochtonen lijken het meest behoefte te hebben naar een adequatere vertegenwoordiging dan na de Tweede Kamerverkiezingen van 2012 het geval was. Echter is het de vraag in hoeverre leden van deze bevolkingsgroepen zich willen committeren aan een strategie. Daarnaast heeft dit onderzoek met enkele beperking te maken gehad als het ging om de gebruikte data. Zo ontbrak data betreffende peilingen onder de bevolkingsgroepen vrouwen, ouderen en provincialen. Vanwege het ontbreken van deze data, is landelijke peiling gebruikt voor deze bevolkingsgroepen. Hierbij zijn we er van uitgegaan dat de landelijke peiling toereikend was voor deze bevolkingsgroepen. De voorspellingen, wat betreft het aantal stemmen dat een kandidaat volgens de peiling zou gaan ontvangen, geven een beter beeld van de werkelijkheid wanneer elke bevolkingsgroep haar/zijn eigen peiling had. Strategie 1, strategie 2 en strategie 4 zijn op de vier verschillende bevolkingsgroepen getest. Betreffende de Tweede Kamerverkiezingen van 2012 geeft dit voldoende validiteit dat de bevolkingsgroepen in hogere mate vertegenwoordigd zouden zijn geweest wanneer zij een strategie zouden hebben uitgevoerd. Echter door deze drie strateg\"{e}n te uit te testen op eerdere verkiezingen kan met grotere zekerheid worden bepaald of de strate\"{e}n daadwerkelijk toereikend kunnen zijn. Desalniettemin is er vooral één test die de daadwerkelijk kracht van een strategie kan aantonen: de komende Tweede Kamerverkiezingen van 15 maart 2017!


\subsection{Acknowledgements}
Me balzak




























\iffalse
\subsection{Voorkeurstemmen en informatiekunde}
\todo{Hier zou ik bescrijven dat toen je hieraan begon je nog dacht dat er flink wat coordinatie nodig zou zijn om tot een maximaal resultaat te komen. En dat die coordinatie dus allicht met een app uitgevoerd zou kunnen worden. Je was van plan verschillende soorten apps tegen elkaar af te zetten (anoniem of niet, persoonlijk of niet, met een cenrale database of niet, etc etc). Uitiendelijk bleek het zo "gemakkelijk" voor een groep om "de kamer te veroveren" dat coordinatie eigenlijk niet echt nodig was. \\
Je kan nog wel wat speculeren over een lagere en sterk varierende en ook onbekende deelname percentage en hoe een app daar zou kunen helpen}

\subsection{Acknowledgements}
Hier kan je bedanken wie je maar wilt.
\fi


























\newpage
