\newpage
\section{Voorkomen Uitbuiting Voorkeursdrempel.}
\label{h8}

\subsection*{Deelvraag 5: Hoe kan er voorkomen worden dat de voorkeursdrempel niet in het voordeel van een bevolkingsgroep kan worden uitgebuit?}

In het vorige hoofdstuk hebben we bewezen dat er een Nash Equilibrium bereikt kan worden wanneer twee bevolkingsgroepen zich beiden committeren aan een strategie. Echter moeten deze bevolkingsgroepen in elke staat op de hoogte zijn van elkaars strategie en daarop inspelen alvorens een Nash Equilibrium kan worden bereikt. In dit hoofdstuk gaan we onderzoeken hoe er een initi\"{e}le staat kan wordt bereikt waardoor de regel van de voorkeursdrempel niet meer uit te buiten is in het voordeel van een bevolkingsgroep. Oftewel, is er een initi\"{e}le staat waarin het Nash Equilibrium al direct is bereikt? Om hierop antwoord te krijgen gebruiken we dezelfde aannames als in het vorige hoofdstuk (zie \hyperref[aannamesNash]{Aannames} in Hoofdstuk \ref{h7}). 


\paragraph{Stelling:} "de staat van een Nash Equilibrium wordt direct bereikt wanneer het aantal stemmen wat nodig is voor het behalen van de voorkeursdrempel gelijk is aan de kiesdeler." \\

Om de bovenstaande stelling te bevestigen of te ontkrachten zullen we hieronder een aantal voorbeelden schetsen. Hierbij moet genoteerd worden dat we de aannames zoals beschreven in het vorige hoofdstuk (zie \hyperref[aannamesNash]{Aannames} in Hoofdstuk \ref{h7}) op één punt veranderen: de voorkeursdrempel wordt in dit hoofdstuk verhoogd van het aantal van dertig stemmen naar het aantal van zestig stemmen. Daarmee is de voorkeursdrempel gelijk aan de kiesdeler. \\
\indent Zoals in het vorige hoofdstuk al is bepaald, kan de Nash-partij tien zetels (\textit{Z=10}) gaan verwachten. Dit komt neer op het aantal van ($10*60$ = ) 600 stemmen. Van de zeshonderd stemmen op de Nash-partij is de kans \textit{P=0,6} dat een stem afkomstig van een vrouwelijke kiezer. Dit komt neer op het aantal van ($0,6*600$ = ) 360 stemmen van vrouwelijke kiezers. Het aantal stemmen dat afkomstig is van mannelijke kiezers komt daarmee uit op ($600-360$ = ) 240 stemmen.\\
\indent Zowel de vrouwelijke als de mannelijke kiezers hanteren in eerste instantie dezelfde strategie. Bij deze strategie worden de stemmen gedeeld voor de voorkeursdrempel (= 60 stemmen) om zodoende de eerste \textit{N} aantal kandidaten op de kandidatenlijst met corresponderende geslacht aan genoeg stemmen te helpen dat zij de voorkeursdrempel behalen. Op deze wijze krijgen de eerste ($360\div60$ = ) 6 kandidaten op de kandidaten op de kandidatenlijst genoeg stemmen om aan de voorkeursdrempel te voldoen. Bij de mannelijke kandidaten krijgen de eerste ($240\div60$ = ) 4 kandidaten op de kandidatenlijst genoeg stemmen om aan de voorkeursdrempel te voldoen. De overige vrouwelijke mannelijke en vrouwelijke kandidaten op de kandidatenlijst ontvangen geen enkele stem. \\
\indent Wanneer mannelijke kiezers er voor kiezen om hun strategie aan te passen om daarmee een voordeel te behalen, kunnen zij hun stemmen verdelen over de eerste drie mannelijke kandidaten op de kandidatenlijst. Op deze wijze komen de eerste drie mannelijke kandidaten op het aantal van ($240\div3$ = ) 80 stemmen per kandidaat. Echter hebben de overige mannelijke kandidaten geen enkele stem ontvangen. Vanwege het feit dat de eerste zes vrouwelijke kandidaten het aantal van zestig stemmen per kandidaat hebben ontvangen, zijn er zes vrouwelijke kandidaten en drie mannelijke kandidaten die een zetels bedeeld krijgen. Zodoende blijft er nog één zetels over. Deze zetel kan voor een mannelijke of vrouwelijke kandidaten zijn. Welke kandidaat deze zetels bedeeld krijgt ligt aan het gegeven of een mannelijke of een vrouwelijke kandidaat op de plaats op de kandidatenlijst staat die recht geeft op de zetel. Echter in het geval de zetel wordt bedeeld aan een mannelijke kandidaat, gaan de mannelijke kiezers er niet op vooruit. Zij hadden immers bij de initi\"{e}le strategie al het aantal van vier zetels behaald en bij deze strategie kunnen zij ook maximaal vier zetels behalen. \\
\indent In het geval de mannelijke kiezers hun strategie niet zouden hebben aangepast en de vrouwelijke kiezers de keuze maken om hun strategie aan te passen, kunnen zij hun stemmen verdelen over de eerste vijf vrouwelijke kandidaten op de kandidatenlijst. Op deze wijze krijgen de eerste vrij vrouwelijke kandidaten op de kandidatenlijst het aantal van ($360\div5$ = ) 72 stemmen per kandidaat. Echter hebben de overige vrouwelijke kandidaten geen enkele stem ontvangen. Zodoende worden de eerste vijf zetels van de Nash-partij bedeeld aan de eerste vijf vrouwelijke kandidaten op de kandidatenlijst. Vanwege het feit dat de eerste vier mannelijke kandidaten 60 stemmen hebben ontvangen per kandidaat worden er vier zetels bedeeld aan mannelijke kandidaten. Nogmaals blijft er zodoende nog één zetel over. Deze zetel kan voor een mannelijke of vrouwelijke kandidaat zijn. En zoals hierboven al beschreven kan deze zetel voor een mannelijke of vrouwelijke kandidaat zijn. De hoogstgeplaatste kandididaat op de kandidatenlijst zonder stemmen krijgt de zetel bedeeld. Wanneer deze zetel wordt bedeeld aan een vrouwelijke kandidaat, gaan de vrouwelijke kiezers er niet op vooruit. Zij hadden immers bij de initi\"{e}le strategie al het aantal van zes zetels behaald en bij deze strategie kunnen zij ook maximaal zes zetels behalen. \\
\indent Zodoende is bij de initi\"{e}le staat al Nash Equilibrium bereikt wanneer beiden partijen een strategie hanteren. De zetelverdeling van zes zetels voor vrouwelijke kandidaten en vier zetels voor mannelijke kandidaten is voor beiden een staat waaruit ze geen voordeel meer kunnen behalen wanneer zij hun initi\"{e}le strategie aanpassen. Hierdoor is verdere uitbuiting van de voorkeursdrempel niet mogelijk. Het aantal zetels wat een bevolkingsgroep krijgt bedeeld is nu een afspiegeling van de verdeling van stemmen van vrouwelijke en mannelijke kiezers op de Nash-partij. 



