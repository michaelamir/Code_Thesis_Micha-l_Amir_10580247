\newpage
\section{Voorkomen Uitbuiting Voorkeursdrempel}
\label{h8}

\subsection*{Deelvraag 5: Hoe kan er voorkomen worden dat de voorkeursdrempel in het voordeel van een bevolkingsgroep kan worden uitgebuit?}

In het vorige hoofdstuk hebben we bewezen dat er een Nash Equilibrium bereikt kan worden wanneer twee bevolkingsgroepen zich beiden committeren aan een strategie. Deze bevolkingsgroepen moeten dan wel op de hoogte zijn van elkaars strategie en daarop inspelen alvorens een Nash Equilibrium kan worden bereikt. In dit hoofdstuk gaan we onderzoeken hoe uitbuiting van de regel van de voorkeursdrempel en daarmee oververtegenwoordiging voorkomen kan worden. Om hier antwoord op te krijgen gebruiken we dezelfde aannames als in het vorige hoofdstuk (zie \hyperref[aannamesNash]{Aannames} in Hoofdstuk \ref{h7}). \\


\begin{theorem}
"De toestand van een Nash Equilibrium wordt direct bereikt wanneer het aantal stemmen dat nodig is voor het behalen van de voorkeursdrempel gelijk is aan de kiesdeler." \\
\end{theorem}

Om de bovenstaande stelling te bevestigen of te ontkrachten zullen we hieronder een aantal voorbeelden schetsen. Hierbij moet genoteerd worden dat we de aannames zoals beschreven in het vorige hoofdstuk (zie \hyperref[aannamesNash]{Aannames} in Hoofdstuk \ref{h7}) op één punt veranderen: de voorkeursdrempel wordt in dit hoofdstuk verhoogd van het aantal van dertig stemmen naar het aantal van zestig stemmen. Daarmee is de voorkeursdrempel gelijk aan de kiesdeler. 

Zoals in het vorige hoofdstuk al is bepaald, kan de Nash-partij tien zetels ($Z=10$) gaan verwachten. Dit komt neer op het aantal van ($10*60$ = ) 600 stemmen. Van de zeshonderd stemmen op de Nash-partij is de kans $P=0,6$ dat een stem afkomstig van een vrouwelijke kiezer. Dit komt neer op het aantal van ($0,6*600$ = ) 360 stemmen van vrouwelijke kiezers. Het aantal stemmen dat afkomstig is van mannelijke kiezers komt daarmee uit op ($600*(1-P)=600-360$ = ) 240 stemmen.

Het scenario dat we gaan bespreken gaat als volgt: zowel de vrouwelijke als de mannelijke kiezers hanteren dezelfde strategie. Bij deze strategie worden de stemmen gedeeld voor de voorkeursdrempel (= 60 stemmen) om zodoende de eerste \textit{N} aantal kandidaten op de kandidatenlijst met corresponderende geslacht aan genoeg stemmen te helpen dat zij de voorkeursdrempel behalen. Op deze wijze krijgen de eerste ($360\div60$ = ) 6 kandidaten op op de kandidatenlijst genoeg stemmen om aan de voorkeursdrempel te voldoen. Bij de mannelijke kandidaten krijgen de eerste ($240\div60$ = ) 4 kandidaten op de kandidatenlijst genoeg stemmen om aan de voorkeursdrempel te voldoen. De overige vrouwelijke mannelijke en vrouwelijke kandidaten op de kandidatenlijst ontvangen geen enkele stem.

Wanneer mannelijke kiezers er voor kiezen om hun strategie aan te passen om daarmee een voordeel te behalen, kunnen zij hun stemmen verdelen over de eerste drie mannelijke kandidaten op de kandidatenlijst. Op deze wijze komen de eerste drie mannelijke kandidaten op het aantal van ($240\div3$ = ) 80 stemmen per kandidaat. De overige mannelijke kandidaten hebben geen enkele stem ontvangen. Vanwege het feit dat de eerste zes vrouwelijke kandidaten het aantal van zestig stemmen per kandidaat hebben ontvangen, zijn er zes vrouwelijke kandidaten en drie mannelijke kandidaten die een zetel bedeeld krijgen. Zodoende blijft er nog één zetels over. Deze zetel kan voor een mannelijke of vrouwelijke kandidaten zijn. Welke kandidaat deze zetels bedeeld krijgt, ligt aan het gegeven of een mannelijke of een vrouwelijke kandidaat op de eerstvolgende plaats op de kandidatenlijst staat die recht geeft op de zetel. Echter in het geval de zetel wordt bedeeld aan een mannelijke kandidaat, gaan de mannelijke kiezers er niet op vooruit. Zij hadden immers bij de initi\"{e}le strategie al het aantal van vier zetels behaald en bij deze strategie kunnen zij ook maximaal vier zetels behalen. 

In het geval de mannelijke kiezers hun strategie niet zouden hebben aangepast en de vrouwelijke kiezers de keuze maken om hun strategie aan te passen, kunnen zij hun stemmen verdelen over de eerste vijf vrouwelijke kandidaten op de kandidatenlijst. Op deze wijze krijgen de eerste vijf vrouwelijke kandidaten op de kandidatenlijst het aantal van ($360\div5$ = ) 72 stemmen per kandidaat. De overige vrouwelijke kandidaten hebben geen enkele stem ontvangen. Zodoende worden de eerste vijf zetels van de Nash-partij bedeeld aan de eerste vijf vrouwelijke kandidaten op de kandidatenlijst. Vanwege het feit dat de eerste vier mannelijke kandidaten 60 stemmen hebben ontvangen per kandidaat, worden er vier zetels bedeeld aan mannelijke kandidaten. Nogmaals blijft er zodoende nog één zetel over. Zoals hierboven al beschreven kan deze zetel voor een mannelijke of vrouwelijke kandidaat zijn. De hoogstgeplaatste kandidaat op de kandidatenlijst zonder stemmen krijgt de zetel bedeeld. Wanneer deze zetel wordt bedeeld aan een vrouwelijke kandidaat, gaan de vrouwelijke kiezers er niet op vooruit. Zij hadden immers bij de initi\"{e}le strategie al het aantal van zes zetels behaald en bij deze strategie kunnen zij ook maximaal zes zetels behalen. 

Zodoende is een Nash Equilibrium direct al bereikt. Ook wanneer beiden partijen een strategie hanteren. De zetelverdeling van zes zetels voor vrouwelijke kandidaten en vier zetels voor mannelijke kandidaten is voor beiden een toestand waaruit ze geen voordeel meer kunnen behalen wanneer zij hun initi\"{e}le strategie aanpassen. Hierdoor is verdere uitbuiting van de voorkeursdrempel niet mogelijk. Het aantal zetels wat een bevolkingsgroep krijgt bedeeld is nu een directe afspiegeling van de verdeling van stemmen van vrouwelijke en mannelijke kiezers op de Nash-partij. 


\paragraph{100\% zekerheid.}
Uit het bovenstaande en het in het vorige hoofdstuk geleverde bewijs kan een regel voor een strategie worden opgesteld die altijd het gewenste rendement zal opleveren ondanks aanpassing van de strategie van de andere bevolkingsgroep(en). Deze regel voor een strategie gaat uit van de volgende aannames:
\begin{itemize}
\item
Alle stemgerechtigden uit de eigen bevolkingsgroep die op de partij willen gaan stemmen doen mee aan de strategie.
\item 
Als bevolkingsgroep zijnde weet je hoeveel de zetels de partij waarop je gaat stemmen zal gaan ontvangen. Daarmee weet je ook het totaal aantal stemmen dat de partij waarop je wil gaan stemmen zal gaan ontvangen.
\item 
Als bevolkingsgroep zijnde weet je het percentage van het totaal aantal stemmen op de partij die afkomstig zijn vanuit de eigen bevolkingsgroep.
\end{itemize}

\noindent Vervolgens luidt de regel: 
\begin{itemize}
\item
De stemmen op de partij afkomstig vanuit de eigen bevolkingsgroep worden willekeurig verdeeld over de top \textit{N} corresponderende kandidaten op de kandidatenlijst van de partij. Hierbij is het top \textit{N} aantal kandidaten een percentage van het totaal aantal te verwachten zetels. Dit percentage is gelijk aan het aandeel stemmen afkomstig vanuit de eigen bevolkingsgroep in verhouding tot het totaal aantal stemmen.
\end{itemize}

Om de hierboven genoemde regel te illustreren nemen we de bevolkingsgroep vrouwen als voorbeeld. Het aantal zetels dat de Nash-partij gaat ontvangen is tien. De stemmen op de Nash-partij zijn verdeeld met 60\% van de stemmen afkomstig van vrouwelijke kiezers en 40\% van de stemmen afkomstig van mannelijke kiezers. Wanneer de vrouwelijke kiezers hun stemmen verdelen over de eerste zes vrouwen op de kandidatenlijst van de Nash-partij krijgen deze vrouwelijke kandidaten met absolute zekerheid het aantal van zes zetels bedeeld, mits de voorkeursdrempel kleiner of gelijk is aan de kiesdeler. Echter lijken de aannames die de regel mogelijk maken niet re\"{e}el. Daarom zullen we in de volgende paragraaf een tweetal adviezen geven die ervoor kunnen zorgen dat een bevolkingsgroep adequater vertegenwoordigd kan worden.

\paragraph*{Adviezen}
In dit hoofdstuk is aangetoond dat de regel van de voorkeursdrempel niet meer is uit te buiten door een bevolkingsgroep wanneer de voorkeursdrempel wordt verhoogd naar hetzelfde aantal stemmen als de kiesdeler. Zodoende kan een bevolkingsgroep niet oververtegenwoordigd zijn wanneer een strategie wordt toegepast. Daarom brengen we het volgende advies uit: om uitbuiting van de regel van de voorkeursdrempel door één of meerdere bevolkingsgroepen te voorkomen, kan de voorkeursdrempel worden verhoogd naar hetzelfde aantal stemmen als het aantal stemmen van de kiesdeler. 

Een tweede advies kan met name een positief gevolg hebben voor de bevolkingsgroep vrouwen. Zoals in Hoofdstuk \ref{h5} al is gesteld lijkt er onder de leden van deze bevolkingsgroep het meeste animo te bestaan voor het uitvoeren van een strategie om zodoende adequater vertegenwoordigd te worden in de Tweede Kamer. Conform de kieswet \citeyearpar{kieswetje} mag het geslacht van de kandidaten op het stemformulier aangeduid worden. Bij de Tweede Kamerverkiezingen van 2012 hadden alle partijen dit ook gedaan \citep{Kiesraad_kandidatenlijsten}. In het geval het eerste advies wordt opgevolgd, luidt het tweede advies als volgt: een eigen stemvakje op het stemformulier voor vrouwelijke kandidaten en een eigen stemvakje voor mannelijke kandidaten. Het is hierbij de bedoeling dat de stemmen op een geslacht worden verdeeld over de top \textit{N} kandidaten op de kandidatenlijst van het corresponderende geslacht. Namelijk, wanneer het eerste advies is opgevolgd en de stemmen niet over de top \textit{N} kandidaten maar over alle kandidaten van het corresponderende geslacht worden verdeeld is het mogelijk dat geen van de kandidaten van dat geslacht boven de voorkeursdrempel komen. Ter illustratie het volgende voorbeeld: vrouwelijke kiezers van de Nash-partij hebben, zoals we weten, 360 stemmen uitgebracht op de partij. Wanneer deze verdeeld worden over de top \textit{N} vrouwelijke kandidaten dat daarmee de voorkeursdrempel kan behalen, zullen er ($360\div60$ = ) 6 vrouwelijke kandidaten een zetels bedeeld krijgen. Echter wanneer de stemmen gelijk verdeeld worden over alle tien de vrouwelijke kandidaten op de kandidatenlijst van de Nash-partij, zullen de vrouwelijke kandidaten, met ($360\div10$ = ) 36 stemmen per kandidaat, allemaal onder de voorkeursdrempel komen. 

Zodoende zullen de stemmen op een geslacht verdeeld dienen te worden over de top \textit{N} kandidaten van het corresponderende geslacht die daarmee de voorkeursdrempel kunnen behalen. Op deze wijze is het aandeel zetels voor kandidaten van een geslacht in verhouding tot het totaal aantal zetels dat de partij ontvangt een directe afspiegeling van de stemverdeling (vrouw-man stemverdeling in het geval van het bovenstaande voorbeeld). Tevens is ook de plaats op de lijst nog altijd van belang. Enkel de top \textit{N} kandidaten van een geslacht zullen een zetel bedeeld krijgen. 

Uiteraard kan op het stemformulier ook worden aangeduid of een kandidaat tot de bevolkingsgroep allochtonen, de bevolkingsgroep ouderen, de bevolkingsgroep provincialen of wat voor bevolkingsgroep dan ook behoort. Verder onderzoek is dan ook nodig om te achterhalen tot in welke mate de verschillende bevolkingsgroepen in Nederland behoefte hebben om adequater vertegenwoordigd te worden in de Tweede Kamer. 


